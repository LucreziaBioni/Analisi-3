
\documentclass{article} % tipologia di documento
\usepackage[utf8]{inputenc}
\usepackage[english]{babel}

\usepackage{ragged2e}
\usepackage[left=25mm, right=25mm, top=15mm]{geometry}
\geometry{a4paper}
\usepackage{graphicx}
\usepackage{booktabs}
\usepackage{paralist}
\usepackage{subfig} 
\usepackage{fancyhdr}
\usepackage{amsmath}
\usepackage{amssymb}
\usepackage{amsfonts}
\usepackage{amsthm}
\usepackage{mathtools}
\usepackage{enumitem}
\usepackage{titlesec}
\usepackage{braket}
\usepackage{gensymb}
\usepackage{url}
\usepackage{hyperref}
\usepackage{csquotes}
\usepackage{multicol}
\usepackage{graphicx}
\usepackage{wrapfig}
\usepackage{babel}
\usepackage{caption}
\captionsetup{font=small}
\pagestyle{fancy}
\renewcommand{\headrulewidth}{0pt}
\lhead{}\chead{}\rhead{}
\lfoot{}\cfoot{\thepage}\rfoot{}
\usepackage{sectsty}
\usepackage[nottoc,notlof,notlot]{tocbibind}
\usepackage[titles,subfigure]{tocloft}
\renewcommand{\cftsecfont}{\rmfamily\mdseries\upshape}
\renewcommand{\cftsecpagefont}{\rmfamily\mdseries\upshape}



\newcommand{\abs}[1]{\left\lvert#1\right\rvert}
\newcommand{\norm}[1]{\left\lVert#1\right\rVert}

\newcommand{\g}{\text{g}}
\newcommand{\m}{\text{m}}
\newcommand{\cm}{\text{cm}}
\newcommand{\mm}{\text{mm}}
\newcommand{\s}{\text{s}}
\newcommand{\N}{\text{N}}
\newcommand{\Hz}{\text{Hz}}

\newcommand{\virgolette}[1]{``\text{#1}"}
\newcommand{\tildetext}{\raise.17ex\hbox{$\scriptstyle\mathtt{\sim}$}}


\renewcommand{\arraystretch}{1.2}

\addto\captionsenglish{\renewcommand{\figurename}{Fig.}}
\addto\captionsenglish{\renewcommand{\tablename}{Tab.}}

\DeclareCaptionLabelFormat{andtable}{#1~#2  \&  \tablename~\thetable}
\title{Formulario di geometria analitica} % titolo del documento
\author{Lucrezia Bioni} % autore del documento
\date{} % data: se è vuoto non mette nulla :)

\begin{document} % inizia il documento
    \maketitle

    \section*{Piani}

    \subsection*{Piano passante per un punto e ortogonale a un vettore} % tra {} titolo del paragrafo
    Un punto $x$ appartiene al piano $\mathcal{P}$ passante per il punto $x_0$ e ortogonale al vettore $\textbf{n}$ se e solo se 
    $x-x_0$ è ortogonale a $\textbf{n}$. Dunque l'equazione vettoriale di $\mathcal{P}$ è \\
    $$
    \langle \ \textbf{n} \ , \ x-x_0  \ \rangle = 0
    $$

    \subsection*{Equazione cartesiana di un piano nello spazio} 
    Ogni piano nello spazio $\mathbb{R}^3$ si rappresenta con un'equazione cartesiana \\
    $$
    ax + by + cz + d = 0 
    $$
    dove almeno uno dei coefficienti $a, b, c$ è non nullo. Viceversa, ogni equazione di questo tipo rappresenta un piano.
    Il vettore $\textbf{v} = (a, b, c)$ è ortogonale al piano di equazione $ax + by + cz + d = 0$, e si chiama \textbf{vettore di giacitura del
    piano}. 
    
    
    \subsection*{Piani paralleli}
    Due piani sono paralleli se hanno la stessa giacitura, cioè se i loro vettori di giacitura $(a , b , c)$ e $( a' , b', c' ) $
    sono proporzionali, cioè se \\
    $$  \exists h \in \mathbb{R} : \ \ \ \ \ \ \ a' = ha \ \ \ \  b'=hb \ \ \ \ c' = hc  $$


    \section*{Rette}

    \subsection*{Retta passante per un punto e parallela a un vettore}
    Retta passante per il punto $P_0$ e parallela al vettore \textbf{v} \\
    $$
    X = P_0 + t \textbf{v} \ \ \ \ \ \ \ \ \ \    t \in \mathbb{R} 
    $$

    \subsection*{Rette parallele}
    Due rette di equazioni parametriche
    sono parallele se hanno la stessa direzione, cioè i se i loro vettori direzione
    $ \textbf{v}=( l, m , n )$ e $ \textbf{v'}=( l', m' , n' ) $ sono proporzionali: 
    $$ \exists h \in \mathbb{R} : \ \ \ \ \ \ \ l' = hl \ \ \ \  m'=hm \ \ \ \ n' = h n  $$ \\


    \section*{Angoli}

    \subsection*{Definzione geometrica}
    Definiamo angolo tra i vettori $\textbf{v}$ e $\textbf{w}$ il numero reale $ \theta \in [0, \pi] $ tale
    $$ cos \theta = \dfrac{ \langle \textbf{v} , \textbf{w} \rangle }{ \| \textbf{v} \| \cdot \| \textbf{w} \| } $$ \\


    \section*{Parallelogrammo}
    \subsection*{Area}
    L'area di un parallelogrammo di lati $\textbf{A}$ e $\textbf{B}$ 
    $$ Area = | \textbf{A} \times \textbf{B} |  $$

\end{document} % fine il documento: non necessario perché lo farà automaticamente