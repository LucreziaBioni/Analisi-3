
\documentclass{article} % tipologia di documento
\usepackage[utf8]{inputenc}
\usepackage[english]{babel}

\usepackage{ragged2e}
\usepackage[left=25mm, right=25mm, top=15mm]{geometry}
\geometry{a4paper}
\usepackage{graphicx}
\usepackage{booktabs}
\usepackage{paralist}
\usepackage{subfig} 
\usepackage{fancyhdr}
\usepackage{amsmath}
\usepackage{amssymb}
\usepackage{amsfonts}
\usepackage{amsthm}
\usepackage{mathtools}
\usepackage{enumitem}
\usepackage{titlesec}
\usepackage{braket}
\usepackage{gensymb}
\usepackage{url}
\usepackage{hyperref}
\usepackage{csquotes}
\usepackage{multicol}
\usepackage{graphicx}
\usepackage{wrapfig}
\usepackage{babel}
\usepackage{caption}
\usepackage{nccmath}
\captionsetup{font=small}
\pagestyle{fancy}
\renewcommand{\headrulewidth}{0pt}
\lhead{}\chead{}\rhead{}
\lfoot{}\cfoot{\thepage}\rfoot{}
\usepackage{sectsty}
\usepackage[nottoc,notlof,notlot]{tocbibind}
\usepackage[titles,subfigure]{tocloft}
\renewcommand{\cftsecfont}{\rmfamily\mdseries\upshape}
\renewcommand{\cftsecpagefont}{\rmfamily\mdseries\upshape}



\newcommand{\abs}[1]{\left\lvert#1\right\rvert}
\newcommand{\norm}[1]{\left\lVert#1\right\rVert}

\newcommand{\g}{\text{g}}
\newcommand{\m}{\text{m}}
\newcommand{\cm}{\text{cm}}
\newcommand{\mm}{\text{mm}}
\newcommand{\s}{\text{s}}
\newcommand{\N}{\text{N}}
\newcommand{\Hz}{\text{Hz}}

\newcommand{\virgolette}[1]{``\text{#1}"}
\newcommand{\tildetext}{\raise.17ex\hbox{$\scriptstyle\mathtt{\sim}$}}


\renewcommand{\arraystretch}{1.2}

\addto\captionsenglish{\renewcommand{\figurename}{Fig.}}
\addto\captionsenglish{\renewcommand{\tablename}{Tab.}}

\DeclareCaptionLabelFormat{andtable}{#1~#2  \&  \tablename~\thetable}
\title{Funzioni implicite} % titolo del documento
\author{Lucrezia Bioni} % autore del documento
\date{} % data: se è vuoto non mette nulla :)

\begin{document} % inizia il documento
    \maketitle

    \subsection*{Teorema di $\exists !$ globale}
    Siano $a<b, \, c<d \, \in \mathbb{R}$ e sia $f: \, [a,b] \times [c,d] \to \mathbb{R}$ continua. Supponiamo che: \\
    $\bullet$ $\forall x \in [a,b]$, $\lim_{y \to c^+} f(x,y)$ e $\lim_{y \to d^-} f(x,y)$ hanno segni discordi \\
    $\bullet$ $\forall (x,y) \in (a,b) \times (c,d)$, $\partial_yf(x,y)$ esiste e ha segno strettamente definito \\
    Allora esiste un'unica funzione $g:(a,b) \to (c,d)$ tale che $f(x , g(x))=0$ per ogni $x \in (a,b)$


    \subsection*{Teorema di Dini, $\exists !$ e regolarità locale}
    Sia U un aperto di $\mathbb{R}^2$ e sia $f:U \to \mathbb{R}$ di classe $\mathcal{C}^1\left(U\right)$. Sia $(x_0,y_0)\in U$ e supponiamo che: \\
    $\bullet$ $f(x_0,y_0)=0$ \\
    $\bullet$ $\partial_y f(x_0,y_0) \neq 0$
    Allora esistono un intorno aperto $V$ di $x_0$, un intorno aperto $W$ di $y_0$ con $V x W \subset U$, ed esiste un'unica funzione $g: V \to W$ tale che: \\
    $\bullet$ $g(x_0)=y_0$, \\
    $\bullet$ $f(x,g(x))=0$ per ogni $x \in V$ \\
    Inoltre $g \in \mathcal{C}^1(V)$ e la sua derivata soddisfa in $V$ l'identità \\
    $g'(x) = - \frac{\partial_x f}{\partial_y f}|_{x,g(x)}$ 
    
    \subsection*{Teorema di $\exists !$ e regolarità locale multi dimensionale}
    Sia U aperto di $\mathbb{R}^{m+n}$ e sia $f:U \to \mathbb{R}^n$ di classe $\mathcal{C}^1\left(U\right)$. Sia $(x_0,y_0) \in U$ e supponiamo che: \\
    $\bullet$ $f(x_0,y_0)=0$ \\
    $\bullet$ det $J_{y}f(x_0,y_0) \neq 0$ \\
    Allora esistono un intorno aperto $V \subset \mathbb{R}^m$ di $x_0$, un intorno aperto $W \subset \mathbb{R}^n$ di $y_0$ con $V x W \subset U$, ed esiste un'unica funzione $g: V \to W$ tale che: \\
    $\bullet$ $g(x_0)=y_0$, \\
    $\bullet$ $f(x,g(x))=0$ per ogni $x \in V$ \\
    Inoltre $g \in \mathcal{C}^1(V)$ e la sua matrice jacobiana soddisfa in $V$ l'identità \\
    $(Jg)|_x = -(J_yf)^{-1}(J_xf)|_{x,g(x)}$ 
    

\end{document} % fine il documento: non necessario perché lo farà automaticamente