
\documentclass{article} % tipologia di documento
\usepackage[utf8]{inputenc}
\usepackage[english]{babel}

\usepackage{ragged2e}
\usepackage[left=25mm, right=25mm, top=15mm]{geometry}
\geometry{a4paper}
\usepackage{graphicx}
\usepackage{booktabs}
\usepackage{paralist}
\usepackage{subfig} 
\usepackage{fancyhdr}
\usepackage{amsmath}
\usepackage{amssymb}
\usepackage{amsfonts}
\usepackage{amsthm}
\usepackage{mathtools}
\usepackage{enumitem}
\usepackage{titlesec}
\usepackage{braket}
\usepackage{gensymb}
\usepackage{url}
\usepackage{hyperref}
\usepackage{csquotes}
\usepackage{multicol}
\usepackage{graphicx}
\usepackage{wrapfig}
\usepackage{babel}
\usepackage{caption}
\captionsetup{font=small}
\pagestyle{fancy}
\renewcommand{\headrulewidth}{0pt}
\lhead{}\chead{}\rhead{}
\lfoot{}\cfoot{\thepage}\rfoot{}
\usepackage{sectsty}
\usepackage[nottoc,notlof,notlot]{tocbibind}
\usepackage[titles,subfigure]{tocloft}
\renewcommand{\cftsecfont}{\rmfamily\mdseries\upshape}
\renewcommand{\cftsecpagefont}{\rmfamily\mdseries\upshape}



\newcommand{\abs}[1]{\left\lvert#1\right\rvert}
\newcommand{\norm}[1]{\left\lVert#1\right\rVert}

\newcommand{\g}{\text{g}}
\newcommand{\m}{\text{m}}
\newcommand{\cm}{\text{cm}}
\newcommand{\mm}{\text{mm}}
\newcommand{\s}{\text{s}}
\newcommand{\N}{\text{N}}
\newcommand{\Hz}{\text{Hz}}

\newcommand{\virgolette}[1]{``\text{#1}"}
\newcommand{\tildetext}{\raise.17ex\hbox{$\scriptstyle\mathtt{\sim}$}}


\renewcommand{\arraystretch}{1.2}

\addto\captionsenglish{\renewcommand{\figurename}{Fig.}}
\addto\captionsenglish{\renewcommand{\tablename}{Tab.}}

\DeclareCaptionLabelFormat{andtable}{#1~#2  \&  \tablename~\thetable}
\title{Teoria del potenziale scalare: campi vettoriali e forme differenziali} % titolo del documento
\author{Lucrezia Bioni} % autore del documento
\date{} % data: se è vuoto non mette nulla :)

\begin{document} % inizia il documento
    \maketitle

    \subsection*{Lavoro del campo lungo una curva}
    Dato un campo vettoriale $\mathcal{F} $ di classe $ \mathcal{C} ^0 $ su $\Omega$ aperto di $ \mathbb{R} ^n $
    e data una curva regolare a tratti $\varphi:[a,b]\to \Omega $, si chiama lavoro del campo lungo la curva data il numero:
    $$ \int _\gamma \langle \mathcal{F} , \tau \rangle \  ds := \int_{a}^{b} \sum_{j = 1}^{n}  \mathcal{F}_j(\varphi (t)) \cdot \varphi_j ' (t) \,dt  $$

    \subsection*{Integrale di una forma differenziale lungo una curva}
    Data $\omega $ forma differenziale su $\Omega $ di classe $ \mathcal{C} ^0 $, $ \omega = \sum_{j = 1}^{n} a_j dx_j $
    e data una curva regolare a tratti $\varphi:[a,b]\to \Omega $, si chiama integrale di $\omega$ lungo $\varphi$ la quantità
    $$ \int _\gamma  \ \omega \ = \ \int _\gamma \ \sum_{j = 1}^{n} \ a_j \ dx_j\ :=\  \int_{a}^{b} \ \sum_{j = 1}^{n} \ a_j ( \varphi (t) ) \ \varphi_j ' (t) \,\ \ dt \ = \ \int _\gamma \ \langle \mathcal{F} _\omega , \tau \rangle \  ds  $$


\end{document} % fine il documento: non necessario perché lo farà automaticamente