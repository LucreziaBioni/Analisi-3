
\documentclass{article} % tipologia di documento
\usepackage[utf8]{inputenc}
\usepackage[english]{babel}

\usepackage{ragged2e}
\usepackage[left=25mm, right=25mm, top=15mm]{geometry}
\geometry{a4paper}
\usepackage{graphicx}
\usepackage{booktabs}
\usepackage{paralist}
\usepackage{subfig} 
\usepackage{fancyhdr}
\usepackage{amsmath}
\usepackage{amssymb}
\usepackage{amsfonts}
\usepackage{amsthm}
\usepackage{mathtools}
\usepackage{enumitem}
\usepackage{titlesec}
\usepackage{braket}
\usepackage{gensymb}
\usepackage{url}
\usepackage{hyperref}
\usepackage{csquotes}
\usepackage{multicol}
\usepackage{graphicx}
\usepackage{wrapfig}
\usepackage{babel}
\usepackage{caption}
\usepackage{nccmath}
\captionsetup{font=small}
\pagestyle{fancy}
\renewcommand{\headrulewidth}{0pt}
\lhead{}\chead{}\rhead{}
\lfoot{}\cfoot{\thepage}\rfoot{}
\usepackage{sectsty}
\usepackage[nottoc,notlof,notlot]{tocbibind}
\usepackage[titles,subfigure]{tocloft}
\renewcommand{\cftsecfont}{\rmfamily\mdseries\upshape}
\renewcommand{\cftsecpagefont}{\rmfamily\mdseries\upshape}



\newcommand{\abs}[1]{\left\lvert#1\right\rvert}
\newcommand{\norm}[1]{\left\lVert#1\right\rVert}

\newcommand{\g}{\text{g}}
\newcommand{\m}{\text{m}}
\newcommand{\cm}{\text{cm}}
\newcommand{\mm}{\text{mm}}
\newcommand{\s}{\text{s}}
\newcommand{\N}{\text{N}}
\newcommand{\Hz}{\text{Hz}}

\newcommand{\virgolette}[1]{``\text{#1}"}
\newcommand{\tildetext}{\raise.17ex\hbox{$\scriptstyle\mathtt{\sim}$}}


\renewcommand{\arraystretch}{1.2}

\addto\captionsenglish{\renewcommand{\figurename}{Fig.}}
\addto\captionsenglish{\renewcommand{\tablename}{Tab.}}

\DeclareCaptionLabelFormat{andtable}{#1~#2  \&  \tablename~\thetable}
\title{Integrazione secondo Lebesgue} % titolo del documento
\author{Lucrezia Bioni} % autore del documento
\date{} % data: se è vuoto non mette nulla :)

\begin{document} % inizia il documento
    \maketitle

    \subsubsection*{Funzione misurabile}
    Sia $f:\mathbb{R} ^n \to \mathbb{R} $, f continua $\implies$ $f$ è misurabile  

    \subsubsection*{Funzione integrabile}
    Diciamo che $f$ è integrabile quando $\int_{E}^{ }f^+\,dx$ e $\int_{E}^{ }f^-\,dx$ sono finiti
    $$ f \in \mathcal{L} (E) \, \, \, \, \, \,  \Longleftrightarrow \, \, \, \, \, \,  f \in \mathfrak{M} (E) \,\, \, \,  \text{e} \,\, \, \,  \int_{E}^{ }|f|\,dx < +\infty  $$


    \subsubsection*{Teorema di convergenza monotona}
    Sia $ \{ f_k \}_{k\in\mathbb{N}}: E \to \overline{\mathbb{R}}, \,\,\, f: E \to \overline{\mathbb{R}}, \,\,\, E \in \mathcal{M} (\mathbb{R} ^n) , \,\,\, f_k \text{ e } f \in \mathfrak{M}(\mathbb{R} ^n) $. \\ Supponiamo che $\exists \, g: E \to \overline{\mathbb{R}}, \,\,\,\,\, g \in \mathcal{L} (E) $, \,\,\,\, con $g(x)\leq f_k(x) \, \,\,\, \forall x \in E \, \,\,\,\, \forall k \in \mathbb{N} $. \\ Supponiamo poi che $\forall x \in E$ $f_k(x)\nearrow f(x)$ (convergenza monotona puntuale)
    $$ \implies f_k \, \, \forall k \,\,\, \text{e} \, f \,\,\, \text{hanno integrale} \,\,\,\, \text{e} \,\,\,\, \lim_{k \to \infty} \int_{E}^{}f_k \,dx = \int_{E}^{}f(x) \,dx $$

    \subsubsection*{Lemma di Fatou}
    Sia $ \ f_k: E \subseteq \mathcal{M} (\mathbb{R} ^n) \to \overline{\mathbb{R}} , \,\,\,\, f_k \in \mathfrak{M} (E) \,\, \forall k$. \\ 1. Supponiamo che $\exists \, g \in \mathcal{L} (E) \,\, , g(x) \leq f_k(x) \,\, \forall x $
    $$ \implies \int_{E}^{}   \lim_{k \to \infty} \text{inf }f_k \,dx  \leq  \lim_{k \to \infty} \text{inf }\int_{E}^{}f_k$$
    2. Supponiamo che $\exists \, G \in \mathcal{L} (E) \,\, , f_k(x) \leq G(x) \,\, \forall x $
    $$ \implies \int_{E}^{}   \lim_{k \to \infty} \text{sup }f_k \,dx  \geq  \lim_{k \to \infty} \text{sup} \int_{E}^{}f_k$$
  

    \subsubsection*{Teorema di convergenza dominata}
    Sia $ \ f_k: E \subseteq \mathcal{M} (\mathbb{R} ^n) \to \overline{\mathbb{R}} , \,\,\,\, f_k \in \mathfrak{M} (E) \,\, \forall k$. Supponiamo che \\ 1. $\exists \, f: E \to \overline{\mathbb{R}} $, con $\lim_{k \to \infty} f_k(x) $ $= f(x)$ $\forall x \in E$ \\ 2. $\exists \, g: E \to \overline{\mathbb{R}} $, $g \in \mathcal{L} (E)$ \,\, con $|f_k(x)| \leq g(x) \,\,\,\,\, \forall x \in E \,\,\,\,\,\, \forall k$
    $$ \implies \lim_{k \to \infty} \int_{E}^{}f_k \,dx = \int_{E}^{}f(x) \,dx $$
  

\end{document} % fine il documento: non necessario perché lo farà automaticamente