
\documentclass{article} % tipologia di documento
\usepackage[utf8]{inputenc}
\usepackage[english]{babel}

\usepackage{ragged2e}
\usepackage[left=25mm, right=25mm, top=15mm]{geometry}
\geometry{a4paper}
\usepackage{graphicx}
\usepackage{booktabs}
\usepackage{paralist}
\usepackage{subfig} 
\usepackage{fancyhdr}
\usepackage{amsmath}
\usepackage{amssymb}
\usepackage{amsfonts}
\usepackage{amsthm}
\usepackage{mathtools}
\usepackage{enumitem}
\usepackage{titlesec}
\usepackage{braket}
\usepackage{gensymb}
\usepackage{url}
\usepackage{hyperref}
\usepackage{csquotes}
\usepackage{multicol}
\usepackage{graphicx}
\usepackage{wrapfig}
\usepackage{babel}
\usepackage{caption}
\usepackage{nccmath}
\captionsetup{font=small}
\pagestyle{fancy}
\renewcommand{\headrulewidth}{0pt}
\lhead{}\chead{}\rhead{}
\lfoot{}\cfoot{\thepage}\rfoot{}
\usepackage{sectsty}
\usepackage[nottoc,notlof,notlot]{tocbibind}
\usepackage[titles,subfigure]{tocloft}
\renewcommand{\cftsecfont}{\rmfamily\mdseries\upshape}
\renewcommand{\cftsecpagefont}{\rmfamily\mdseries\upshape}



\newcommand{\abs}[1]{\left\lvert#1\right\rvert}
\newcommand{\norm}[1]{\left\lVert#1\right\rVert}

\newcommand{\g}{\text{g}}
\newcommand{\m}{\text{m}}
\newcommand{\cm}{\text{cm}}
\newcommand{\mm}{\text{mm}}
\newcommand{\s}{\text{s}}
\newcommand{\N}{\text{N}}
\newcommand{\Hz}{\text{Hz}}

\newcommand{\virgolette}[1]{``\text{#1}"}
\newcommand{\tildetext}{\raise.17ex\hbox{$\scriptstyle\mathtt{\sim}$}}


\renewcommand{\arraystretch}{1.2}

\addto\captionsenglish{\renewcommand{\figurename}{Fig.}}
\addto\captionsenglish{\renewcommand{\tablename}{Tab.}}

\DeclareCaptionLabelFormat{andtable}{#1~#2  \&  \tablename~\thetable}
\title{%
    \Huge Formulario \\
    \Large Analisi Matematica 3}
\author{Lucrezia Bioni} % autore del documento
\date{} % data: se è vuoto non mette nulla :)

\begin{document} % inizia il documento
    \maketitle

    \section{Funzioni Implicite}
    \subsection{Teoremi}
    \subsubsection*{Teorema di $\exists !$ globale}
    Siano $a<b, \, c<d \, \in \mathbb{R}$ e sia $f: \, [a,b] \times [c,d] \to \mathbb{R}$ continua. Supponiamo che: \\
    $1.$ $\forall x \in [a,b]$, $\lim_{y \to c^+} f(x,y)$ e $\lim_{y \to d^-} f(x,y)$ hanno segni discordi \\
    $2.$ $\forall (x,y) \in (a,b) \times (c,d)$, $\partial_yf(x,y)$ esiste e ha segno strettamente definito \\
    Allora esiste un'unica funzione $g:(a,b) \to (c,d)$ tale che $f(x , g(x))=0$ per ogni $x \in (a,b)$


    \subsubsection*{Teorema di Dini, $\exists !$ e regolarità locale}
    Sia U un aperto di $\mathbb{R}^2$ e sia $f:U \to \mathbb{R}$ di classe $\mathcal{C}^1\left(U\right)$. Sia $(x_0,y_0)\in U$ e supponiamo che: \\
    $1.$ $f(x_0,y_0)=0$ \\
    $2.$ $\partial_y f(x_0,y_0) \neq 0$
    Allora esistono un intorno aperto $V$ di $x_0$, un intorno aperto $W$ di $y_0$ con $V x W \subset U$, ed esiste un'unica funzione $g: V \to W$ tale che: \\
    $1.$ $g(x_0)=y_0$, \\
    $2.$ $f(x,g(x))=0$ per ogni $x \in V$ \\
    Inoltre $g \in \mathcal{C}^1(V)$ e la sua derivata soddisfa in $V$ l'identità $g'(x) = - \frac{\partial_x f}{\partial_y f}|_{x,g(x)}$ 
    
    \subsubsection*{Teorema di $\exists !$ e regolarità locale multi dimensionale}
    Sia U aperto di $\mathbb{R}^{m+n}$ e sia $f:U \to \mathbb{R}^n$ di classe $\mathcal{C}^1\left(U\right)$. Sia $(x_0,y_0) \in U$ e supponiamo che: \\
    $1.$ $f(x_0,y_0)=0$ \\
    $2.$ det $J_{y}f(x_0,y_0) \neq 0$ \\
    Allora esistono un intorno aperto $V \subset \mathbb{R}^m$ di $x_0$, un intorno aperto $W \subset \mathbb{R}^n$ di $y_0$ con $V x W \subset U$, ed esiste un'unica funzione $g: V \to W$ tale che: \\
    $1.$ $g(x_0)=y_0$, \\
    $2.$ $f(x,g(x))=0$ per ogni $x \in V$ \\
    Inoltre $g \in \mathcal{C}^1(V)$ e la sua matrice jacobiana soddisfa in $V$ l'identità \\
    $(Jg)|_x = -(J_yf)^{-1}(J_xf)|_{x,g(x)}$ 

    \subsection{Definizioni}
    \subsubsection*{Retta tangente all'implicita}
    Sia $y=f(x)$ definita implicitamente. Per trovare il piano tangente in $(x_0,y_0)$:
    $f'(x)=-\frac{\partial_x F}{\partial_y F}$ \\
    $0 = \partial{xx}F + 2 \partial_{xy} F f' + \partial_{yy} F  (f')^2 + \partial_y F f''$
    \subsubsection*{Piano tangente all'implicita 2D}
    Sia $z=f(x,y)$ definita implicitamente. 

    \section{Forme differenziali}
    \subsection{Teoremi e definizioni}
    \subsubsection*{Lavoro del campo lungo una curva}
    Dato un campo vettoriale $\mathcal{F} $ di classe $ \mathcal{C} ^0 $ su $\Omega$ aperto di $ \mathbb{R} ^n $
    e data una curva regolare a tratti $\varphi:[a,b]\to \Omega $, si chiama lavoro del campo lungo la curva data il numero:
    $$ \int _\gamma \langle \mathcal{F} , \tau \rangle \  ds := \int_{a}^{b} \sum_{j = 1}^{n}  \mathcal{F}_j(\varphi (t)) \cdot \varphi_j ' (t) \,dt  $$

    \subsubsection*{Integrale di una forma differenziale lungo una curva}
    Data $\omega $ forma differenziale su $\Omega $ di classe $ \mathcal{C} ^0 $, $ \omega = \sum_{j = 1}^{n} a_j dx_j $
    e data una curva regolare a tratti $\varphi:[a,b]\to \Omega $, si chiama integrale di $\omega$ lungo $\varphi$ la quantità
    $$ \int _\gamma  \ \omega \ = \ \int _\gamma \ \sum_{j = 1}^{n} \ a_j \ dx_j\ :=\  \int_{a}^{b} \ \sum_{j = 1}^{n} \ a_j ( \varphi (t) ) \ \varphi_j ' (t) \,\ \ dt \ = \ \int _\gamma \ \langle \mathcal{F} _\omega , \tau \rangle \  ds  $$

    \subsubsection*{Teorema di caratterizzazione}
    Data $\omega$ forma differenziale su $\Omega$ (aperto di $\mathcal{R}^n$ ) di classe $\mathcal{C} ^n$. \\
    I seguenti fatti sono equivalenti: \\
    $\bullet \, \omega$ è esatta in $\Omega$ \\
    $\bullet \int_{\gamma} \omega = 0 \forall$ curva $\gamma$ regolare (a tratti) e chiusa in $\Omega$ \\
    $\bullet$ $\forall p,q$ in $\Omega$, comunque si prenda una curva regolare (a tratti) in $\Omega$ da p a q e orientata (da p a q) si ha che $ \int_{\text{curva da p a q}} \omega$ dipende solo da p e q, ma non dipende dalla curva $\gamma$


    \subsubsection*{Forma differenziale chiusa}
    Sia $\omega = a_1 dx_1 + ... + a_n dx_n$ una forma differenziale di classe $\mathcal{C} ^1$ su $\Omega$ aperto di $\mathbb{R}^n$, $\omega$ è detta chiusa \\
    $$ \frac{\partial a_i}{ \partial x_j} = \frac{\partial a_j}{\partial x_i} \,\,\, \forall i,j $$

    \subsubsection*{Campo vettoriale non rotazionale}
    Sia $F = (F_1,...,F_n)$ un campo vettoriale di classe $\mathcal{C} ^1$ su $\Omega$, aperto di $\mathbb{R}^n$, F è detto non rotazionale (o irrotazionale) quando si verifica:
    $$\frac{\partial F_i}{ \partial x_j} = \frac{\partial F_j}{\partial x_i} \,\,\, \forall i,j$$

    \subsubsection*{Proposizione}
    Sia $\omega$ una forma differenziale di classe $\mathcal{C}^1$ su $\Omega$ aperto di $\mathbb{R}^n$.
    Se $\omega$ è esatta in $\Omega$ $\implies$ $\omega$ è chiusa in $\Omega$.

    \subsubsection*{Lemma di Poincaré}
    Sia $\omega$ una forma differenziale di classe $\mathcal{C}^1$ su $\Omega$ aperto di $\mathbb{R}^n$. \\
    Se $\Omega$ è stellato e $\omega$ è chiusa in $\Omega$ $\implies$ $\omega$ è esatto in $\Omega$.\\
    Valido anche nel caso in cui $\Omega$ sia semplicemente connesso.

    \subsubsection*{Omotopia}
    Siano $\phi_0, \phi_1:[0,1] \to \Omega \subseteq \mathbb{R}^n$. \\ Supponiamo che $\phi_0$ e $\phi_1$ siano curve con $\phi_0(0)=\phi_1(0)$ e $\phi_0(1)=\phi_1(1)$. \\ Le due curve $\phi_0$ e $\phi_1$ sono dette omotope quando esiste una mappa (mappa di omotopia) $\psi:[o,1] \times [0,1] \to \Omega$ continua globalmente e tale che: \\
    $\bullet \, \psi(0,t)=\phi_0(t) \forall t$ \\
    $\bullet \, \psi(s,0)=\phi_1(t) \forall t$ \\
    $\bullet \, \psi(s,0)$ non dipende da s e $\psi(s,0)$ non dipende da s.

    \subsubsection*{Teorema di invarianza omotopica}
    Sia $\omega$ una forma differenziale su $\Omega$, aperto di $\mathbb{R}^n$, di classe $\mathcal{C}^1$. Supponiamo che $\omega$ sia chiusa in $\Omega$. \\
    Siano $\phi_0$ e $\phi_1$ curve regolari a tratti da p a q in $\Omega$. \\
    Se $\phi_0$ e $\phi_1$ sono omotope $\implies$ $\int_{\phi_0} \omega = \int_{\phi_1} \omega$

    \section{Ottimizzazione}
    \subsection*{Ottimizzazione vincolata}
    \subsubsection*{Teorema dei moltiplicatori di Lagrange}
    Sia $f:\Omega \subseteq \mathbb{R} ^n \to \mathbb{R}$, $\Omega$ aperto, $f \in \mathcal{C} ^1 (\Omega)$. \\ Sia $D \subseteq \Omega$ l'insieme degli zeri di una mappa $F: \Omega \to \mathbb{R} ^m \, (m<n), F \in \mathcal{C} ^1$. \\ Supponiamo che $x_0 \in D$ sia un estremo locale per f ristretto a D. \\ Supponiamo che $J_f{x_0}$ abbia rango massimo (ovvero di rango m). \\
    Allora $\exists$ $\lambda_1,...,\lambda_m \in \mathbb{R}$ t.c. $\nabla f(x_0) = \lambda_1 \nabla F_1(x_0) + ... + \lambda_m \nabla F_m(x_0)$, dove $F=(F_1, ... , F_m)$.


    \section{Integrazione secondo Lebesgue}
    \section{Teoremi}
    \subsubsection*{Funzione misurabile}
    Sia $f:\mathbb{R} ^n \to \mathbb{R} $, f continua $\implies$ $f$ è misurabile  

    \subsubsection*{Funzione integrabile}
    Diciamo che $f$ è integrabile quando $\int_{E}^{ }f^+\,dx$ e $\int_{E}^{ }f^-\,dx$ sono finiti
    $$ f \in \mathcal{L} (E) \, \, \, \, \, \,  \Longleftrightarrow \, \, \, \, \, \,  f \in \mathfrak{M} (E) \,\, \, \,  \text{e} \,\, \, \,  \int_{E}^{ }|f|\,dx < +\infty  $$

    \subsubsection*{Condizione sufficiente di integrabilità}
    $\bullet$ \, Se f è misurabile e limitata e $m(E) < + \infty$ $\implies f \in L(E)$ \\
    $\bullet$ \, $f \in \mathcal{R} (I), I=[a,b] \implies f \in L(I)$ \\
    $\bullet$ \, $f: I \to \mathbb{R}$, con I intervallo (non necessariamente compatto). Supponiamo che $f$ sia assolutamente integrabile secondo Riemann generalizzato in I $\implies f \in L(I)$

    \subsubsection*{Teorema di convergenza uniforme}
    Sia $E \in \mathcal{M}(\mathbb{R}^n)$, con $m(E) < + \infty$. \\ Se $\{f_k\}_{k \in \mathbb{N}}: E \to \mathbb{R}, f_k \in L(E)$, $f:E \to \mathbb{R}$, e $f_k \to f$ uniformemente in E \\
    Allora $f \in L(E)$ e, inoltre, $\lim_{k \to \infty} \int_E \left| f_k(x) - f(x) \right|dx=0$, così che $\lim_{k \to \infty} \int_E f_k$ esiste e vale $\int_E f$


    \subsection*{Teorema di convergenza monotona}
    Sia $ \{ f_k \}_{k\in\mathbb{N}}: E \to \overline{\mathbb{R}}, \,\,\, f: E \to \overline{\mathbb{R}}, \,\,\, E \in \mathcal{M} (\mathbb{R} ^n) , \,\,\, f_k \text{ e } f \in \mathfrak{M}(\mathbb{R} ^n) $. \\ Supponiamo che $\exists \, g: E \to \overline{\mathbb{R}}, \,\,\,\,\, g \in \mathcal{L} (E) $, \,\,\,\, con $g(x)\leq f_k(x) \, \,\,\, \forall x \in E \, \,\,\,\, \forall k \in \mathbb{N} $. \\ Supponiamo poi che $\forall x \in E$ $f_k(x)\nearrow f(x)$ (convergenza monotona puntuale) \\
    $ \implies f_k \, \, \forall k \,\,\, \text{e} \, f \,\,\, \text{hanno integrale} \,\,\,\, \text{e} \,\,\,\, \lim_{k \to \infty} \int_{E}^{}f_k \,dx = \int_{E}^{}f(x) \,dx $

    \subsection*{Lemma di Fatou}
    Sia $ \ f_k: E \subseteq \mathcal{M} (\mathbb{R} ^n) \to \overline{\mathbb{R}} , \,\,\,\, f_k \in \mathfrak{M} (E) \,\, \forall k$. \\ 1. Supponiamo che $\exists \, g \in \mathcal{L} (E) \,\, , g(x) \leq f_k(x) \,\, \forall x $
    $ \implies \int_{E}^{}   \lim_{k \to \infty} \text{inf }f_k \,dx  \leq  \lim_{k \to \infty} \text{inf }\int_{E}^{}f_k$ \\
    2. Supponiamo che $\exists \, G \in \mathcal{L} (E) \,\, , f_k(x) \leq G(x) \,\, \forall x $
    $ \implies \int_{E}^{}   \lim_{k \to \infty} \text{sup }f_k \,dx  \geq  \lim_{k \to \infty} \text{sup} \int_{E}^{}f_k$
  

    \subsection*{Teorema di convergenza dominata}
    Sia $ \ f_k: E \subseteq \mathcal{M} (\mathbb{R} ^n) \to \overline{\mathbb{R}} , \,\,\,\, f_k \in \mathfrak{M} (E) \,\, \forall k$. Supponiamo che \\ 1. $\exists \, f: E \to \overline{\mathbb{R}} $, con $\lim_{k \to \infty} f_k(x) $ $= f(x)$ $\forall x \in E$ \\ 2. $\exists \, g: E \to \overline{\mathbb{R}} $, $g \in \mathcal{L} (E)$ \,\, con $|f_k(x)| \leq g(x) \,\,\,\,\, \forall x \in E \,\,\,\,\,\, \forall k$
    $ \implies \lim_{k \to \infty} \int_{E}^{}f_k \,dx = \int_{E}^{}f(x) \,dx $
  

    \section{Integrazione multidimensionale}

    \section{Teoremi e definizioni}
    
    \subsubsection*{Teorema di Fubini}
    Sia $f:E \subseteq \mathbb{R}^n \to \mathbb{R}$, $E \in \mathcal{M}(\mathbb{R}^n), f \in L(E)$. Sia $\mathbb{R}^m \times \mathbb{R}^k = \mathbb{R}^n$ una decomposizione ortogonale. Allora: \\
    $\bullet \,$ per $q.o. \, x$ la sezione $E(x) = \{y \in \mathbb{R}^k$ con $(x,y) \in E \}$ è misurabile in $\mathbb{R}^k$ \\
    $\bullet \,$ per $q.o \, $ la funzione $x \mapsto \int_{E(x)} f(x,y)dy$ è ben definita, ed è in $L(\mathbb{R}^m)$ \\
    $\bullet \,$ $\int_E f(x,y) dx dy = \int_{\mathbb{R}^m} \left[\int_{E(x)} f(x,y) dy\right] dx$
  
    \subsubsection*{Teorema di Tonelli}
    Sia $f:E \subseteq \mathbb{R}^n \to \mathbb{R}$, $E \in \mathcal{M}(\mathbb{R}^n), f \in \mathcal{M}(E), f(x) \geq 0 \, \forall x$. Sia $\mathbb{R}^m \times \mathbb{R}^k = \mathbb{R}^n$ una decomposizione ortogonale. Allora: \\
    $\bullet \,$ per $q.o. \, x$ la sezione $E(x) = \{y \in \mathbb{R}^k$ con $(x,y) \in E \}$ è misurabile in $\mathbb{R}^k$ \\
    $\bullet \,$ per $q.o \, x$ la funzione $x \mapsto \int_{E(x)} f(x,y)dy$ è ben definita, ed è in $L(\mathbb{R}^m)$ \\
    $\bullet \,$ $\int_E f(x,y) dx dy = \int_{\mathbb{R}^m} \left[\int_{E(x)} f(x,y) dy\right] dx$


    \subsubsection*{Teorema per il cambiamento di coordinate}
    Sia $\Phi:\Omega \subseteq \mathbb{R}^n \to \tilde{\Omega} $, con $\Omega$ e $\tilde{\Omega}$ aperti, un cambiamento di coordinate (dunque un diffeomorfismo). Sia $E \subseteq \tilde{\Omega}$, $E \in \mathcal{M}(\mathbb{R})$ $\implies \Phi^{-1}(E) \in \mathcal{M}(\mathbb{R}^n)$.\\
    Sia $f:E \to \mathbb{R}, f \in L(E)$ oppure $f:E \to [0,+\infty]$ e misurabile. Allora:
    $$\int_{\Phi^{-1}(E)}f(\Phi (x)) \left| \text{det} J \Phi (x) \right| dx = \int_E f(y) dy$$

    \subsubsection*{Formule di Green}
    Sia D un dominio regolare in $\mathbb{R}^2$. Sia $f: \Omega \subseteq \mathbb{R}^2 \to \mathbb{R}, D \subseteq \Omega, \Omega$ aperto, $f \in \mathcal{C}^1(\Omega)$, allora: \\
    $$\int_D \frac{\partial f}{\partial x}(x,y) dx dy = \int_{\partial D^+} f(x,y) dy$$
    $$\int_D \frac{\partial f}{\partial y}(x,y) dx dy = - \int_{\partial D^+} f(x,y) dx$$

    \subsubsection*{Teorema della divergenza (Gauss) e di Stokes in $\mathbb{R}^2$}
    Sia $D \subseteq \mathbb{R}^2$ un dominio regolare. Sia $F: \Omega \to \mathbb{R} ^2, D \subseteq \Omega, \Omega$ aperto, un campo vettoriale di classe $\mathcal{C}^1(\Omega)$, $F=(f,g)$ allora: \\
    $$ \int_D \left( \frac{\partial f}{ \partial x} + \frac{\partial g}{ \partial y} \right) dx dy = \int_D \text{Div} F dx dy = \int_{\partial D^+} (f dy - g dx) = \int_{\partial D ^+} < F , \nu > ds $$
    $$ \int_D \left( \frac{\partial g}{ \partial x} - \frac{\partial f}{ \partial y} \right) dx dy = \int_{\partial D^+} (f dx + g dy) = \int_{\partial D ^+} < F , \tau > ds $$

    \subsubsection*{Equazione del piano tangente}
    Sia $s_0 \in \overset{\circ}{S} \left(\equiv \text{Im} \phi\left(\overset{\circ}{D}\right)\right)$. Allora S ha in $s_0$ un piano tangente che ha equazione:
    $$ \text{det} \left[{ \mathbf{x}} - s_0 \, | \, \partial_u \phi (u_0, v_0) \, | \, \partial_v \phi (u_0, v_0) \right] = 0$$

    \subsubsection*{Integrale di superficie}
    Data una superficie $\phi$ (non necessariamente orientabile) di sostegno S, sia $f: W \subseteq \mathbb{R}^3 \to \mathbb{R}$, con S $\subseteq W$, dico che f è integrabile (secondo Lebesgue) su S quando $f \circ \phi \, \lVert \mathbf{ \partial_u \phi \wedge \partial_v \phi} \rVert$ è Lebesgue integrabile in D. In tal caso si pone: 
    $$\int_S f \, d \sigma = \int_D (f \circ \phi) (u,v) \, \lVert \mathbf{ \partial_u \phi \wedge \partial_v \phi} \rVert \, du dv $$

    \subsubsection*{Teorema della divergenza (Gauss) in $\mathbb{R}^3$}
    Sia T un dominio regolare e sia $\mathbf{F}=(F_1, F_2, F_3)$ un campo vettoriale di classe $\mathcal{C}^1(\Omega)$, con $\Omega$ aperto, $T \subseteq \Omega$, allora:
    $$\int_T Div \mathbf{F} \,\, dx \, dy \, dz = \int_{\partial T^+} <\mathbf{F}, \mathbf{\nu}> \, d\sigma$$
    Dove $ Div \mathbf{F} = \frac{\partial F_1}{\partial x} + \frac{\partial F_2}{\partial y} + \frac{\partial F_3}{\partial z}$ e $\nu = \frac{ \mathbf{ \partial_u \phi \wedge \partial_v \phi}}{\lVert \mathbf{ \partial_u \phi \wedge \partial_v \phi} \rVert}$

    \subsubsection*{Teorema di Stokes in $\mathbb{R}^3$}
    Sia $\phi:D \subseteq \mathbb{R}^2 \to \mathbb{R}^3$ una superficie regolare con bordo, con D dominio regolare, di sostegno S. Sia $\mathbf{F}$ un campo vettoriale di classe $\mathcal{C}^1(W)$ con W un aperto di $\mathbb{R}^3, S \subseteq W$. Allora:
    $$\int_S < \text{rot}F, \nu> \, d \sigma = \int_{\partial S^+} < \mathbf{F} , \tau > \, ds$$
    Dove rot$\mathbf{F}$:
    $$\begin{bmatrix}
        i & \partial x & F_1\\
        j & \partial y & F_2 \\
        k & \partial z & F_3 \\
    \end{bmatrix}$$

    \subsection{Domini notevoli in $\mathbb{R}^3$}
    \subsubsection*{Paraboloide}
    $z = z_0 + x^2 + y^2$ - Paraboloide con vertice in $(0,0,z_0)$ e aperto verso + \\
    $y = y_0 + x^2 + z^2$ - Paraboloide su asse y \\
    $x = x_0 + y^2 + z^2$ - Paraboloide su asse x \\
    Se è della forma $z=-(x^2 + y^2)$ è ribaltato.
    \subsubsection*{Sfera}
    $(x-x_0)^2+(y-y_0)^2+(z-z^0)^2=r^2$ - Sfera di raggio r e centro $(x_0,y_0,z_0)$
    \subsubsection*{Cono}
    $z = \sqrt{x^2+y^2}$ - Cono standard \\
    $z = z_0 \pm \sqrt{x^2+y^2}$ - Cono di vertice $(0,0,z_0)$. Se segno - è ribaltato.
    \subsubsection*{Cilindro}
    $x^2+y^2=c^2$ - Cilindro di raggio c.
    \subsubsection*{Iperboloide}
    $z^2-(x^2+y^2)=c^2$ - Iperboloide di vertice c.
    \subsubsection*{Sfera parziale}
    $z= a \pm \sqrt{b-x^2-y^2}$ - Porzione della sfera ridotta a un intorno del polo N/S

\end{document} % fine il documento: non necessario perché lo farà automaticamente