
\documentclass{article} % tipologia di documento
\usepackage[utf8]{inputenc}
\usepackage[english]{babel}

\usepackage{ragged2e}
\usepackage[left=25mm, right=25mm, top=15mm]{geometry}
\geometry{a4paper}
\usepackage{graphicx}
\usepackage{booktabs}
\usepackage{paralist}
\usepackage{subfig} 
\usepackage{fancyhdr}
\usepackage{amsmath}
\usepackage{amssymb}
\usepackage{amsfonts}
\usepackage{amsthm}
\usepackage{mathtools}
\usepackage{enumitem}
\usepackage{titlesec}
\usepackage{braket}
\usepackage{gensymb}
\usepackage{url}
\usepackage{hyperref}
\usepackage{csquotes}
\usepackage{multicol}
\usepackage{graphicx}
\usepackage{wrapfig}
\usepackage{babel}
\usepackage{caption}
\usepackage{nccmath}
\captionsetup{font=small}
\pagestyle{fancy}
\renewcommand{\headrulewidth}{0pt}
\lhead{}\chead{}\rhead{}
\lfoot{}\cfoot{\thepage}\rfoot{}
\usepackage{sectsty}
\usepackage[nottoc,notlof,notlot]{tocbibind}
\usepackage[titles,subfigure]{tocloft}
\renewcommand{\cftsecfont}{\rmfamily\mdseries\upshape}
\renewcommand{\cftsecpagefont}{\rmfamily\mdseries\upshape}



\newcommand{\abs}[1]{\left\lvert#1\right\rvert}
\newcommand{\norm}[1]{\left\lVert#1\right\rVert}

\newcommand{\g}{\text{g}}
\newcommand{\m}{\text{m}}
\newcommand{\cm}{\text{cm}}
\newcommand{\mm}{\text{mm}}
\newcommand{\s}{\text{s}}
\newcommand{\N}{\text{N}}
\newcommand{\Hz}{\text{Hz}}

\newcommand{\virgolette}[1]{``\text{#1}"}
\newcommand{\tildetext}{\raise.17ex\hbox{$\scriptstyle\mathtt{\sim}$}}


\renewcommand{\arraystretch}{1.2}

\addto\captionsenglish{\renewcommand{\figurename}{Fig.}}
\addto\captionsenglish{\renewcommand{\tablename}{Tab.}}

\DeclareCaptionLabelFormat{andtable}{#1~#2  \&  \tablename~\thetable}
\title{Ottimizzazione} % titolo del documento
\author{Lucrezia Bioni} % autore del documento
\date{} % data: se è vuoto non mette nulla :)

\begin{document} % inizia il documento
    \maketitle
    \section*{Ottimizzazione libera}
    Aggiungo teoremi Analisi 2.

    \section*{Ottimizzazione vincolata}
    \subsection*{Teorema dei moltiplicatori di Lagrange}
    Sia $f:\Omega \subseteq \mathbb{R} ^n \to \mathbb{R}$, $\Omega$ aperto, $f \in \mathcal{C} ^1 (\Omega)$. \\ Sia $D \subseteq \Omega$ l'insieme degli zeri di una mappa $F: \Omega \to \mathbb{R} ^m \, (m<n), F \in \mathcal{C} ^1$. \\ Supponiamo che $x_0 \in D$ sia un estremo locale per f ristretto a D. \\ Supponiamo che $J_f{x_0}$ abbia rango massimo (ovvero di rango m). \\
    Allora $\exists$ $\lambda_1,...,\lambda_m \in \mathbb{R}$ t.c. $\nabla f(x_0) = \lambda_1 \nabla F_1(x_0) + ... + \lambda_m \nabla F_m(x_0)$, dove $F=(F_1, ... , F_m)$.
    
    

\end{document} % fine il documento: non necessario perché lo farà automaticamente