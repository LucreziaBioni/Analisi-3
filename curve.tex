
\documentclass{article} % tipologia di documento
\usepackage[utf8]{inputenc}
\usepackage[english]{babel}

\usepackage{ragged2e}
\usepackage[left=25mm, right=25mm, top=15mm]{geometry}
\geometry{a4paper}
\usepackage{graphicx}
\usepackage{booktabs}
\usepackage{paralist}
\usepackage{subfig} 
\usepackage{fancyhdr}
\usepackage{amsmath}
\usepackage{amssymb}
\usepackage{amsfonts}
\usepackage{amsthm}
\usepackage{mathtools}
\usepackage{enumitem}
\usepackage{titlesec}
\usepackage{braket}
\usepackage{gensymb}
\usepackage{url}
\usepackage{hyperref}
\usepackage{csquotes}
\usepackage{multicol}
\usepackage{graphicx}
\usepackage{wrapfig}
\usepackage{babel}
\usepackage{caption}
\usepackage{nccmath}
\captionsetup{font=small}
\pagestyle{fancy}
\renewcommand{\headrulewidth}{0pt}
\lhead{}\chead{}\rhead{}
\lfoot{}\cfoot{\thepage}\rfoot{}
\usepackage{sectsty}
\usepackage[nottoc,notlof,notlot]{tocbibind}
\usepackage[titles,subfigure]{tocloft}
\renewcommand{\cftsecfont}{\rmfamily\mdseries\upshape}
\renewcommand{\cftsecpagefont}{\rmfamily\mdseries\upshape}



\newcommand{\abs}[1]{\left\lvert#1\right\rvert}
\newcommand{\norm}[1]{\left\lVert#1\right\rVert}

\newcommand{\g}{\text{g}}
\newcommand{\m}{\text{m}}
\newcommand{\cm}{\text{cm}}
\newcommand{\mm}{\text{mm}}
\newcommand{\s}{\text{s}}
\newcommand{\N}{\text{N}}
\newcommand{\Hz}{\text{Hz}}

\newcommand{\virgolette}[1]{``\text{#1}"}
\newcommand{\tildetext}{\raise.17ex\hbox{$\scriptstyle\mathtt{\sim}$}}


\renewcommand{\arraystretch}{1.2}

\addto\captionsenglish{\renewcommand{\figurename}{Fig.}}
\addto\captionsenglish{\renewcommand{\tablename}{Tab.}}

\DeclareCaptionLabelFormat{andtable}{#1~#2  \&  \tablename~\thetable}
\title{Curve e integrali curvilinei} % titolo del documento
\author{Lucrezia Bioni} % autore del documento
\date{} % data: se è vuoto non mette nulla :)

\begin{document} % inizia il documento
    \maketitle

    \subsection*{Curva parametrizzata in parametro d'arco}
    Data una curva $\varphi$ regolare, $\varphi$ è detta curva parametrizzata in parametro d'arco
    quando $ \| \varphi  '(t) \| = 1 \ \ \ \ \ \forall t $

    \subsection*{Lunghezza della curva}
    Data una curva $\varphi:[a,b]\to \mathbb{R} ^n $ di classe $ \mathcal{C} ^1([a,b]) $ chiamo 
    lunghezza di $\varphi$ il numero 
    $$ \mathcal{L} _\varphi := \int_{a}^{b} \| \varphi  '(t) \| \, dt  $$

    \subsection*{Triedro fondamentale (di Frenet/moving frame)}
    Data una curva $\varphi:[a,b]->\mathbb{R} ^3$ , $\varphi \in \mathcal{C} ^2 $ regolare

    \subsubsection*{Parametro d'arco}
    Riparametrizzo per lunghezza d'arco (= arclength o lunghezza curvilinea): \\ Trovo $s(t) = \int_{0}^{t} \left| \varphi'(t) \right| \,dt $. \\ Trovo $t(s)$ invertendo la relazione precedente e ottengo $\tilde{\varphi}(s) = \varphi( t(s) )$.
    
    \subsubsection*{ 1. Versore tangente}
    \underline{Parametro d'arco $s$} \\ Approssima la curva al primo ordine. Per la parametrizzazione scelta, è già di lunghezza 1. $$T(s)=\frac{d\tilde{\varphi}}{ds} $$ \\ \underline{Generico parametro $t$} $$T(t) = \frac{d\varphi}{dt} \frac{dt}{ds}$$

    \subsubsection*{ 2. Versore normale}
    \underline{Parametro d'arco $s$} $$T(s)=\frac{ \frac{dT}{ds}}{ \left| \frac{dT}{ds} \right|} $$ \\ \underline{Generico parametro $t$} $$T(t)=\frac{ \frac{dT}{dt}}{ \left| \frac{dT}{dt} \right|}$$

    \subsubsection*{ 3. Versore binormale}
    $$ B(s) = T \times N $$

    \subsection*{Formule di Frenet-Serret}
    La base ortonormale ($T, N, B$) soddisfa il seguente sistema di ODE:
    \begin{equation}
        \begin{cases}
            \frac{dT}{ds} = k N \\
            \frac{dN}{ds} = -kT + \tau B \\
            \frac{dB}{ds} = - \tau N
        \end{cases}\,
    \end{equation}
    Dove \\ \textbf{Curvatura della curva} $k = \left| \frac{dT}{ds} \right| $ \\ \textbf{Torsione della curva} $ \tau = - \langle \frac{dB}{ds} , N \rangle , | \tau | = | \frac{dB}{ds} |  $
    

\end{document} % fine il documento: non necessario perché lo farà automaticamente